\section{并行编程和计算思维}
到目前为止,我们专注于并行编程的实践知识,包括 CUDA 编程接口特性、GPU 架构、性能优化技术、并行模式和应用案例研究。 在本章中,我们将讨论转向更抽象的概念。 我们将并行编程概括为设计或选择并行算法并将领域问题分解为明确定义和协调的工作单元的计算思维过程,每个工作单元都可以由所选算法有效地执行。 具有强大计算思维能力的程序员不仅可以分析领域问题的结构,还可以转换领域问题的结构:哪些部分本质上是串行的,哪些部分适合高性能并行执行,以及将部分从 前者为后者。 通过良好的算法选择和问题分解,程序员可以在并行性、工作效率和资源消耗之间实现适当的折衷。 为具有挑战性的领域问题创建成功的计算解决方案通常需要领域知识和计算思维技能的强有力结合。 本章将使读者更深入地了解并行编程和一般计算思维。

\subsection{并行编程的目标}