\section{排序}
排序算法将列表的数据元素按一定的顺序排列。 排序是现代数据和信息服务的基础,因为如果数据集顺序正确,则可以显着降低从数据集中检索信息的计算复杂性。 例如,排序通常用于规范化数据,以便在数据列表之间进行快速比较和协调。 此外,如果数据按一定顺序排列,则可以提高许多数据处理算法的效率。 由于其重要性,高效排序算法一直是许多计算机科学研究的主题。 即使使用这些高效的算法,对大型数据列表进行排序仍然很耗时,并且可以从并行执行中受益。 并行化高效排序算法具有挑战性,需要精心设计。 本章介绍两种重要的高效排序算法的并行设计:基数排序和合并排序。 本章的大部分内容致力于基数排序; 基于第 12 章“合并”中介绍的并行合并模式,简要讨论了合并排序。 还简要讨论了其他流行的并行排序算法,例如转置排序和采样排序。

\subsection{背景}