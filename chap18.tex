\section{静电势图}
前面的案例研究使用统计估计应用程序来说明选择适当级别的循环嵌套以进行并行执行、转换循环以减少内存访问干扰、使用恒定内存来放大只读数据的内存带宽、使用 寄存器来减少内存带宽的消耗,并使用特殊的硬件功能单元来加速三角函数。 在本案例研究中,我们使用基于常规网格数据结构的分子动力学应用程序来说明如何使用优化技术来实现全局内存访问、合并和提高计算吞吐量。 正如我们在之前的案例研究中所做的那样,我们提出了一系列静电势图计算内核的实现,其中每个版本都对前一个版本进行了改进。 每个版本都采用第 6 章“性能注意事项”中的一种或多种实用技术。 前面的案例研究中使用了一些技术,但有些技术有所不同:计算结果的系统重用、线程粒度粗化和快速边界条件检查。 该应用案例研究表明,有效使用这些实用技术可以显着提高应用程序的执行吞吐量。

\subsection{背景}