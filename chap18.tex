\section{静电势图}
前面的案例研究使用统计估计应用程序来说明选择适当级别的循环嵌套以进行并行执行、转换循环以减少内存访问干扰、使用恒定内存来放大只读数据的内存带宽、使用 寄存器来减少内存带宽的消耗,并使用特殊的硬件功能单元来加速三角函数。 在本案例研究中,我们使用基于常规网格数据结构的分子动力学应用程序来说明如何使用优化技术来实现全局内存访问、合并和提高计算吞吐量。 正如我们在之前的案例研究中所做的那样,我们提出了一系列静电势图计算内核的实现,其中每个版本都对前一个版本进行了改进。 每个版本都采用第 6 章“性能注意事项”中的一种或多种实用技术。 前面的案例研究中使用了一些技术,但有些技术有所不同:计算结果的系统重用、线程粒度粗化和快速边界条件检查。 该应用案例研究表明,有效使用这些实用技术可以显着提高应用程序的执行吞吐量。

\subsection{背景}
本案例研究基于视觉分子动力学 (VMD)(Humphrey 等人,1996),这是一种流行的软件系统,旨在显示、动画和分析生物分子系统。 VMD拥有超过20万注册用户。 它是现代“计算显微镜”的重要基础,生物学家可以用它来观察微小的生命形式,例如对于传统显微镜技术来说太小的病毒。 虽然它对分析生物分子系统有强大的内置支持,例如计算分子系统空间网格点的静电势值(本章的重点),但它也是显示其他大型数据集的流行工具,例如 测序数据、量子化学模拟数据和体积数据,由于其多功能性和用户可扩展性。

虽然 VMD 设计为在各种硬件上运行,包括笔记本电脑、台式机、集群和超级计算机,但大多数用户将 VMD 用作交互式三维 (3D) 可视化和分析的桌面科学应用程序。 对于交互使用时运行时间过长的计算,VMD 还可以以批处理模式使用来渲染电影以供以后使用。 加速 VMD 的一个动机是使批处理模式作业足够快以供交互使用。 这可以极大地提高科学研究的生产力。 随着 CUDA 设备在台式电脑中广泛使用,这种加速可以对 VMD 用户社区产生广泛的影响。 迄今为止,VMD 的多个方面已通过 CUDA 得到加速,包括静电势图计算、离子放置、分子轨道计算和显示以及蛋白质中气体迁移路径的成像。

本案例研究涵盖的计算是网格空间中静电势图的计算。 此计算通常用于将离子放置到分子结构中以进行分子动力学模拟。 图 18.1 显示了离子在蛋白质结构中的位置,为分子动力学模拟做准备。 在此应用中,静电势图用于根据物理定律识别离子(红点)可以适应的空间位置。 该函数还可用于计算分子动力学模拟过程中的时均电场电位图,这对于模拟过程以及模拟结果的可视化和分析很有用。

有多种计算静电势图的方法。 其中,直接库仑求和(DCS)是一种高精度方法,特别适合GPU(Stone et al., 2007)。 DCS方法将每个网格点的静电势值计算为系统中所有原子的贡献之和。 如图 18.2 所示。 原子 i 对晶格点 j 的贡献是原子 i 的电荷除以晶格点 j 到原子 i 的距离。 由于这需要对所有网格点和所有原子进行,因此计算次数与系统中原子总数和网格点总数的乘积成正比。 对于现实的分子系统,该乘积可能非常大。 因此,静电势图的计算传统上是作为 VMD 中的批处理作业完成的。

\subsection{核函数设计中的分散与聚集}
图 18.3 显示了 DCS 代码的基本 C 代码。 该函数被编写为处理 3D 网格的二维 (2D) 切片。 该函数将为建模空间的所有切片重复调用。 该函数的结构非常简单,只有三层 for 循环。 外部两个级别迭代网格点空间的 y 维度和 x 维度。 对于每个网格点,最里面的 for 循环迭代所有原子,计算所有原子对网格点的静电势能的贡献。 请注意,每个原子由atoms[] 数组的四个连续元素表示。 前三个元素存储原子的 x、y 和 z 坐标,第四个元素存储原子的电荷。 在最内层循环结束时,将网格点的累加值写入网格数据结构。 然后,外部循环迭代并将执行执行到下一个网格点。

请注意,图 18.3 中的 DCS 函数通过将网格点索引值乘以网格点之间的间距来动态计算每个网格点的 x 和 y 坐标。 这是一种均匀网格方法,其中所有网格点在所有三个维度上都以相同的距离间隔。 该函数利用了同一切片中的所有网格点具有相同 z 坐标的事实。 该值由函数的调用者预先计算并作为函数参数 (z) 传入。 然而,可以对图 18.3 中的顺序 C 代码进行一些优化,以显着提高其执行速度。

图 18.4 显示了 DCS 的 C 代码,经过一些优化以提高其执行速度和效率。 首先,图 18.3 中的最内循环(n 循环)已交换为最外循环(图 18.4 中的第 05 行)。 因此,代码会迭代所有原子。 对于每个原子,内部循环(i 循环和 j 循环)将原子的贡献分散到所有网格点。 正如我们在第 17 章“迭代磁共振成像重建”中讨论的,循环互换是允许的,因为图 18.3 中的三层循环是完美嵌套的,并且所有迭代都是相互独立的。

循环交换实现了两种优化。 首先,原子与平面中所有网格点之间的距离的 z 分量是相同的,并且可以针对整个网格点切片计算一次。 因此,计算可以在两个内部循环之外完成(第 6-7 行)。 类似地,原子与同一行中所有网格点之间距离的 y 分量是相同的,并且可以在最内层循环之外完成(第 11-12 行)。 相比之下,距离的 y 和 z 分量都是在图 18.3 中最里面的循环中计算的。 计算次数的大幅减少使得图 18.4 中的 C 代码速度更快。 这些优化无法在图 18.3 中完成,因为最里面的循环会迭代所有原子,因此当原子之间的变化时,必须重新计算距离的 x、y 和 z 分量。

对于 GPU 执行,我们假设主机程序在系统内存中输入并维护原子电荷及其坐标。 它还在系统内存中维护网格点数据结构。 DCS 内核设计用于处理静电势网格点结构的 2D 切片(不要与线程网格混淆)。 这些网格点类似于第 8 章“模板”中讨论的离散化网格点。 对于每个 2D 切片,CPU 将其网格数据传输到设备全局内存。 与 k 空间数据类似(第 17 章,迭代磁共振成像重建),原子信息被分成块以适合常量存储器。 对于原子信息的每个chunk,CPU将该chunk传输到设备常量内存中,调用DCS内核计算当前chunk对当前分片的贡献,并准备传输下一个chunk。 当前切片的所有原子信息块处理完毕后,该切片被传回以更新CPU系统内存中的网格点数据结构。 然后系统移动到下一个切片。

现在让我们重点讨论DCS内核的设计。 对图 18.4 中优化的 C 代码进行并行化是很自然的。 生成的内核如图 18.5 所示。 定义的常量 CHUNK\_SIZE 指定每次内核调用应传输到 GPU 常量内存中的原子数量。 CHUNK\_SIZE × 4 的值应小于或等于 64K。 内核使用每个线程来实现图 18.4 中最外层循环的迭代,并将其分配的原子的贡献分散到所有网格点。 不幸的是,正如我们在第 17 章“迭代磁共振成像重建”中了解到的那样,这种分散的并行化方法需要原子操作来更新能量网格点(第 17-18 行),这显着降低了并行执行的速度。

正如我们在第 17 章“迭代磁共振成像重建”中了解到的,我们可以使用一种聚集方法,其中每个线程计算所有原子对一个网格点的累积贡献。 这是一种首选方法,因为每个线程都将写入自己的网格点,并且不需要使用原子操作。 然而,这需要按照图18.3中未优化的C代码的顺序来排列循环; 也就是说,我们将并行化较慢的 C 实现。 这体现了并行化应用程序中经常遇到的困境:优化的顺序代码不像未优化的顺序代码那样适合并行化。 缺点是每个线程内的执行速度可能会大大减慢,这会降低并行化的速度优势。 我们将在本章后面回到这一点。

图 18.6 显示了基于收集方法的内核。 内核基于图 18.3 中未优化的 C 代码。 我们形成一个与 2D 潜在网格点组织相匹配的 2D 线程网格。 为此,我们需要将图 18.3 的第 04-06 行中的两个外部循环修改为完美的嵌套循环,以便我们可以使用每个线程执行两级循环的一次迭代。 我们可以执行循环裂变(就像我们在之前的案例研究中所做的那样)或将 y 坐标(图 18.3 的第 05 行)的计算移到内部循环中。 前者需要我们创建一个新数组来保存所有 y 值,并导致两个内核通过全局内存通信数据。 后者增加了 y 坐标的计算次数。 在这种情况下,我们选择执行后者,因为只有少量计算可以轻松容纳在内循环中,而不会显着增加内循环的执行时间。 吸收到内循环中的工作量比第 17 章“迭代磁共振成像重建”中的工作量小得多。 前者会增加线程执行很少工作的内核的内核启动开销。 所选转换允许并行执行所有 i 和 j 迭代。 这是完成的计算量和实现的并行性水平之间的权衡。

在图 18.6 的内核代码内部,图 18.3 中循环的外部两级已被删除,并被内核调用中的执行配置参数取代(图 18.6 的第 04-05 行)。 在每个线程网格内,组织线程块来计算网格结构的图块的静电电位。 在最简单的内核中,每个线程计算一个网格点的值。 在更复杂的内核中,每个线程计算多个网格点,并利用网格点计算之间的冗余来提高执行速度。 这是第 6 章“性能注意事项”中讨论的线程粗化优化的示例,并将在下一节中讨论。

图 18.6 中的内核的性能非常好,因为它的执行速度不受原子操作的阻碍。 此外,快速浏览一下代码就会发现,每个线程对每访问四个内存元素执行九次浮点运算。 每个原子的这些atoms[]数组元素被缓存在每个流式多处理器(SM)中的硬件常量高速缓存中,并且被广播到许多线程。 跨线程大量重用这些常量内存元素使得常量缓存极其有效,消除了绝大多数 DRAM 访问。 因此,全局内存带宽不是该内核的限制因素。