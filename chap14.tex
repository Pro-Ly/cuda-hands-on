\section{稀疏矩阵计算}
我们的下一个并行模式是稀疏矩阵计算。 在稀疏矩阵中,大多数元素为零。 存储和处理这些零元素在内存容量、内存带宽、时间和能量方面都是浪费的。 许多重要的现实问题都涉及稀疏矩阵计算。 由于这些问题的重要性,几种稀疏矩阵存储格式及其相应的处理方法被提出并在该领域得到广泛应用。 所有这些方法都采用某种类型的压缩技术来避免存储或处理零元素,但代价是在数据表示中引入某种程度的不规则性。 不幸的是,这种不规则性可能导致并行计算中内存带宽的利用不足、控制流发散和负载不平衡。 因此,在压缩和正则化之间取得良好的平衡非常重要。 一些存储格式在高度不规则性的情况下实现了更高水平的压缩。 其他人实现了更适度的压缩水平,同时保持表示更规则。 众所周知,使用每种存储格式的并行计算的相对性能在很大程度上取决于稀疏矩阵中非零元素的分布。 了解稀疏矩阵存储格式及其相应并行算法的大量工作,为并行程序员在解决相关问题时应对压缩和正则化挑战提供了重要的背景。

\subsection{背景}