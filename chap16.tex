\section{深度学习}
本章介绍了深度学习的应用案例研究,深度学习是使用人工神经网络的机器学习的最新分支。 机器学习已被用于许多应用领域,根据从数据集中收集的经验来训练或调整应用程序逻辑。 为了有效,人们通常需要使用大量数据进行此类训练。 虽然机器学习作为计算机科学的一个学科已经存在很长时间了,但由于两个原因,它最近在实际行业中获得了广泛的认可。 第一个原因是互联网的普遍使用提供了大量数据。 第二个原因是廉价的大规模并行 GPU 计算系统,可以利用这些海量数据集有效地训练应用程序逻辑。 我们将从机器学习和深度学习的简要介绍开始,然后更详细地考虑最流行的深度学习算法之一:卷积神经网络(CNN)。 CNN 具有较高的计算与内存访问比率和高水平的并行性,这使它们成为 GPU 加速的完美候选者。 我们将首先介绍卷积神经网络的基本实现。 接下来,我们将展示如何使用共享内存改进这个基本实现。 然后,我们将展示如何将卷积层表示为矩阵乘法,这可以通过在现代 GPU 中使用高度优化的硬件和软件来加速。

\subsection{背景}