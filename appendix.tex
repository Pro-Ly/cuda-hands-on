\section{数值考量}
在计算的早期,浮点运算能力仅存在于大型机和超级计算机中。 尽管许多20世纪80年代设计的微处理器开始配备浮点协处理器,但它们的浮点运算速度极慢,比大型机和超级计算机慢大约三个数量级。 随着微处理器技术的进步,许多 20 世纪 90 年代设计的微处理器,例如 Intel Pentium III 和 AMD Athlon,开始具备可与超级计算机相媲美的高性能浮点功能。 高速浮点运算是当今微处理器和 GPU 的标准功能。 浮点表示允许可表示数据值的更大动态范围和微小数据值的更精确表示。 这些理想的特性使浮点运算成为物理和人工现象建模的首选数据代表,例如燃烧、空气动力学、光照和金融风险。 这些模型的大规模评估一直在推动对并行计算的需求。 因此,应用程序员在开发并行应用程序时了解浮点运算的本质非常重要。 我们将特别关注浮点算术运算的准确性、浮点数表示的精度、数值算法的稳定性以及在并行编程中应如何考虑这些因素。

\subsection{浮点数表示}