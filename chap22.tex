\section{高级实践与未来演进}
本书的重点是可扩展的并行编程。 CUDA C 和 GPU 硬件在我们的示例和练习中主要扮演了编程平台的角色。 然而,在CUDA C基础上学习的并行编程概念和技能可以很容易地适应其他并行编程平台。 例如,正如我们在第 20 章“异构计算集群编程”、“异构计算集群编程:CUDA 流简介”中看到的,消息传递接口 (MPI) 的大多数关键概念(例如进程、等级和屏障)都有对应的概念 此外,正如我们在第 20 章“异构计算集群编程”中讨论的那样,支持 CUDA 的 GPU 已在高性能计算 (HPC) 系统中广泛使用。 对于许多读者来说,CUDA C 可能是一个重要的应用程序开发和部署平台,而不仅仅是一个学习工具。 因此,读者了解旨在支持应用程序级别高性能编程的高级 CUDA C 功能和实践非常重要。 例如,正如我们在第 20 章“异构计算集群编程”中看到的,CUDA 流使 MPI HPC 应用程序能够与计算重叠通信。 这种能力对于实现整个应用程序的性能目标尤其重要。 考虑到这一点,本章将为读者概述 CUDA C 和 GPU 计算硬件的高级功能,这些功能对于在应用程序中实现高性能和可维护性非常重要。 对于每个功能,我们将介绍基本概念以及其在不同代 GPU 计算中的演变简史。 充分理解这些概念和演变历史将有助于消除对这些功能的一些常见困惑。 目标是帮助读者建立一个概念框架,以便更详细地研究这些功能。

\subsection{主机/设备交互模型}