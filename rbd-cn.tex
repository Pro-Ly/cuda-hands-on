%!TEX encoding = UTF-8 Unicode
\documentclass{ctexart}
\usepackage{color}
\usepackage{xcolor}
\usepackage{amsmath}
\usepackage{amssymb}
\usepackage{esint}
\usepackage{graphicx}
\usepackage{bm}
\usepackage{multirow}

\numberwithin{equation}{section}

\newtheorem{Definition}{\hspace{2em}定义}
\newtheorem{theorem}{\hspace{2em}定理}
\newtheorem{lemma}{\hspace{2em}引理}
\newtheorem{Proof}{证明}
\newtheorem{remark}{注}
\newtheorem{example}{例子}

\colorlet{RED}{red}

\begin{document}

\title{刚体动力学中的算法}

\author{杨丰}
\date{}
\maketitle
\section*{前言}
本书的目的是提供大量用于计算刚体动力学的最有效算法,并对它们进行足够详细的解释,使读者能够理解它们的工作原理,
以及如何调整它们(或创建新算法) 以满足读者的需要。 
该集合包括以下著名算法:递归牛顿-欧拉算法、复合刚体算法和关节体算法。 它还包括用于运动学环路和浮动基台的算法。 
每个算法都源自第一性原理,并以一组方程式和伪代码程序的形式呈现,后者旨在轻松翻译成任何合适的编程语言。

本书还解释了一些用于制定刚体系统运动方程的数学技巧。 特别是,它展示了如何使用六维 (6D) 向量表达动力学,
并解释了作为高效算法基础的递归公式。 其他主题包括:如何构建刚体系统的计算机模型; 
利用惯性矩阵中的稀疏性; 关节体惯性的概念; 动力学计算中舍入误差的来源; 以及刚体之间物理接触和冲击的动力学。

刚体动力学有成为代数海洋的趋势。 然而,这在很大程度上是使用 3D 向量的结果,可以通过使用 6D 向量表示法来补救。 
本书使用基于空间向量的记号,其中刚体运动的线性和角度方面被组合成一组统一的量和方程。 
结果通常是代数量减少四到六倍。 计算机代码也能感受到这种好处:更短、更清晰的程序更易于阅读、编写和调试,
但仍然与使用标准 3D 矢量的代码一样高效。

本书旨在面向广大读者,从高年级本科生到研究人员和专业人士。 
假定读者具有一些刚体动力学的先验知识,例如可以从动力学入门课程或阅读介绍性文本的前几章中获得的知识。 
然而,不需要 6D 向量的先验知识,因为这个主题在第 2 章的开头进行了解释。
本书还包含一些高级材料, 6D 向量的动力学专家和学者可能会感兴趣。 
本书不附带任何软件,但读者可以从作者的网站上获得此处描述的大多数算法的源代码。

本书原本打算成为 1987 年出版的名为《机器人动力学算法》的书的第二版; 
但很快就很清楚,有足够的新材料来证明写一本全新的书是合理的。 
与其前身相比,这里最引人注目的新材料是:算法的显式伪代码描述; 
关于如何为刚体系统建模的一章; 利用分支引起的稀疏性的算法; 
运动学循环和浮动基础系统的扩展处理; 平面矢量(空间矢量的平面等价物); 
数值误差和模型敏感性; 以及如何实现空间矢量算法的指南。

\newpage
\tableofcontents


\end{document}


























