\section{有序合并:动态输入数据识别介绍}
我们的下一个并行模式是有序合并操作,它采用两个排序列表并生成一个组合的排序列表。 有序合并操作可以用作排序算法的构建块,我们将在第 13 章“排序”中看到。 有序合并操作也构成了现代映射缩减框架的基础。 本章介绍了一种并行有序合并算法,其中每个线程的输入数据是动态确定的。 数据访问的动态特性使得利用局部性和平铺技术来提高内存访问效率和性能变得具有挑战性。 动态输入数据识别背后的原理也与许多其他重要计算相关,例如集合交集和集合并集。 我们提出了日益复杂的缓冲区管理方案,以提高顺序合并和其他动态确定其输入数据的操作的内存访问效率。

\subsection{背景}