\section{有序合并:动态输入数据识别介绍}
我们的下一个并行模式是有序合并操作,它采用两个排序列表并生成一个组合的排序列表。 有序合并操作可以用作排序算法的构建块,我们将在第 13 章“排序”中看到。 有序合并操作也构成了现代映射缩减框架的基础。 本章介绍了一种并行有序合并算法,其中每个线程的输入数据是动态确定的。 数据访问的动态特性使得利用局部性和平铺技术来提高内存访问效率和性能变得具有挑战性。 动态输入数据识别背后的原理也与许多其他重要计算相关,例如集合交集和集合并集。 我们提出了日益复杂的缓冲区管理方案,以提高顺序合并和其他动态确定其输入数据的操作的内存访问效率。

\subsection{背景}

\subsection{总结}
在本章中,我们介绍了有序合并模式,其并行化要求每个线程动态识别其输入位置范围。 由于输入范围与数据相关,因此我们采用 co-rank 函数的快速搜索实现来识别每个线程的输入范围。 当我们使用平铺技术来节省内存带宽并启用内存合并时,输入范围依赖于数据的事实也带来了额外的挑战。 因此,我们引入了循环缓冲区的使用,以允许我们充分利用从全局内存加载的数据。 我们表明,引入更复杂的数据结构(例如循环缓冲区)可以显着增加使用该数据结构的代码的复杂性。 因此,我们为操作和使用索引的代码引入了一个简化的缓冲区访问模型,以保持基本不变。 仅当这些索引用于访问缓冲区中的元素时,缓冲区的实际循环性质才会暴露。