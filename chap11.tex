\section{前缀和(扫描):并行算法工作效率介绍}
我们的下一个并行模式是前缀和,通常也称为扫描。 并行扫描经常用于并行化看似顺序的操作,例如资源分配、工作分配和多项式求值。 一般来说,如果计算自然地被描述为数学递归,其中系列中的每个项目都是根据前一个项目来定义的,那么它很可能被并行化为并行扫描操作。 并行扫描在大规模并行计算中发挥着关键作用,原因很简单:应用程序的任何顺序部分都会极大地限制应用程序的整体性能。 许多这样的连续部分可以通过并行扫描转换为并行计算。 因此,并行扫描通常用作并行算法中的基本操作,执行基数排序、快速排序、字符串比较、多项式求值、求解递归、树操作和流压缩。 基数排序示例将在第 13 章“排序”中介绍。

并行扫描是一种重要的并行模式的另一个原因是,它是一些并行算法执行的工作可能比顺序算法执行的工作具有更高复杂性的典型示例,从而导致需要在 算法复杂度和并行化。 正如我们将要展示的,算法复杂性的轻微增加可能会使并行扫描比大型数据集的顺序扫描运行得更慢。 在“大数据”时代,这种考虑变得更加重要,因为海量数据集对计算复杂度较高的传统算法提出了挑战。

\subsection{背景}