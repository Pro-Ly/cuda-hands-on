\section{并行直方图:原子操作和私有化简介}
到目前为止,我们提出的并行计算模式都允许将计算每个输出元素的任务专门分配给线程或由线程拥有。 因此,这些模式符合所有者计算规则,其中每个线程都可以写入其指定的输出元素,而无需担心其他线程的干扰。 本章介绍并行直方图计算模式,其中每个输出元素都可以由任何线程更新。 因此,在更新输出元素时必须注意线程之间的协调,并避免任何可能破坏最终结果的干扰。 在实践中,还有许多其他重要的并行计算模式,其中输出干扰无法轻易避免。 因此,并行直方图算法提供了这些模式中发生的输出干扰的示例。 我们将首先检查使用原子操作序列化每个元素的更新的基线方法。 这种基线方法简单但效率低下,常常导致执行速度令人失望。 然后,我们将介绍一些广泛使用的优化技术,尤其是私有化,以显着提高执行速度,同时保持正确性。 这些技术的成本和收益取决于底层硬件以及输入数据的特征。 因此,对于开发人员来说,理解这些技术的关键思想并能够推理它们在不同情况下的适用性非常重要。

\subsection{背景}