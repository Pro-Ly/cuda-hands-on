\section{CUDA 动态并行}
CUDA 动态并行性是 CUDA 编程模型的扩展,它使内核能够调用其他内核,从而允许在设备上执行的线程启动新的线程网格。 在 CUDA 的早期版本中,网格只能从主机代码启动。 涉及递归、不规则循环结构、时空变化或其他不适合平面和单级并行性的结构的算法需要通过主机的多个内核调用来实现,这增加了主机的负担、数量 主机设备通信的数量以及总执行时间。 在某些情况下,程序员诉诸循环序列化和其他笨拙的技术来支持这些算法需求,但以软件可维护性为代价。 对动态并行性的支持允许动态发现新工作的算法来准备和启动新网格,而不会增加主机负担或影响软件可维护性。 本章介绍了支持动态并行性的 CUDA 扩展功能,包括对 CUDA 编程接口的修改和添加,以及利用这种附加功能的指南和最佳实践。

\subsection{背景}