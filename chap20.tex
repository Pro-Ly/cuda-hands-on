\section{异构计算集群编程:CUDA stream 简介}
到目前为止,我们专注于对一台主机和一台设备的异构计算系统进行编程。 在高性能计算(HPC)中,应用程序需要计算节点集群的聚合计算能力。 如今,许多 HPC 集群的每个节点都有一台或多台主机和一台或多台设备。 从历史上看,这些集群主要使用消息传递接口 (MPI) 进行编程。 在本章中,我们将介绍 MPI/CUDA 联合编程。 我们将仅介绍程序员需要了解的 MPI 概念,以将其异构应用程序扩展到集群环境中的多个节点。 特别是,我们将重点关注将 CUDA 内核扩展到多个节点的背景下的域分区、点对点通信和集体通信。

\subsection{背景}