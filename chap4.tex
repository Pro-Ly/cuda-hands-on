\section{计算架构和调度}
在第 1 章“简介”中,我们看到 CPU 的设计目的是最大限度地减少指令执行的延迟,而 GPU 的设计目的是最大限度地提高执行指令的吞吐量。 在第 2 章“异构数据并行计算”和第 3 章“多维网格和数据”中,我们学习了 CUDA 编程接口的核心功能,用于创建和调用内核来启动和执行线程。 在接下来的三章中,我们将讨论现代 GPU 的架构,包括计算架构和内存架构,以及源于对这种架构的理解的性能优化技术。 本章介绍了 GPU 计算架构的几个方面,这些方面对于 CUDA C 程序员理解和推理其内核代码的性能行为至关重要。 我们将首先展示计算架构的高级简化视图,并探讨灵活的资源分配、块调度和占用的概念。 然后我们将深入讨论线程调度、延迟容忍、控制发散和同步。 我们将通过描述可用于查询 GPU 中可用资源的 API 函数以及在执行内核时帮助估计 GPU 占用率的工具来结束本章。 在接下来的两章中,我们将介绍GPU内存架构的核心概念和编程注意事项。 特别是,第 5 章“内存架构和数据局部性”重点介绍了片上内存架构,第 6 章“性能考虑因素”简要介绍了片外内存架构,然后详细阐述了整个 GPU 架构的各种性能考虑因素。 掌握这些概念的 CUDA C 程序员能够很好地编写和理解高性能并行内核。

\subsection{现代GPU架构}
图 4.1 显示了 CUDA C 程序员对典型支持 CUDA 的 GPU 架构的高级视图。 它被组织成一系列高度线程化的流式多处理器(SM)。 每个 SM 都有多个称为流处理器或 CUDA 核心(以下简称为核心)的处理单元,如图 4.1 中 SM 内的小块所示,它们共享控制逻辑和内存资源。 例如,Ampere A100 GPU有108个SM,每个SM有64个核心,整个GPU总共有6912个核心。

SM 还具有不同的片上存储器结构,在图 4.1 中统称为“存储器”。 这些片上存储器结构将是第 5 章“存储器架构和数据局部性”的主题。 GPU 还配备了千兆字节的片外设备内存,在图 4.1 中称为“全局内存”。 虽然较旧的 GPU 使用图形双数据速率同步 DRAM,但从 NVIDIA Pascal 架构开始的较新 GPU 可能会使用 HBM(高带宽内存)或 HBM2,它们由与 GPU 紧密集成在同一个内存中的 DRAM(动态随机存取内存)模块组成。 包裹。 为了简洁起见,在本书的其余部分中,我们将所有这些类型的内存广泛地称为 DRAM。 我们将在第 6 章“性能注意事项”中讨论访问 GPU DRAM 所涉及的最重要概念。