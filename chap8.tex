\section{模板 Stencil}
模板是流体动力学、热导、燃烧、天气预报、气候模拟和电磁学等应用领域中求解偏微分方程数值方法的基础。 基于模板的算法处理的数据由具有物理意义的离散量组成,例如质量、速度、力、加速度、温度、电场和能量,它们之间的关系由微分方程控制。 模板的常见用途是根据输入变量值范围内的函数值来近似函数的导数值。 模板与卷积非常相似,因为模板和卷积都根据同一位置的元素的当前值以及另一个多维数组中邻域中的元素的当前值来计算多维数组的元素的新值。 因此模板还需要处理晕细胞和鬼细胞。 与卷积不同,模板计算用于迭代求解感兴趣域内连续、可微函数的值。 用于模板邻域中的元素的数据元素和权重系数由正在求解的微分方程控制。 某些模板图案适合不适用于卷积的优化。 在初始条件通过域迭代传播的求解器中,输出值的计算可能具有依赖性,并且需要根据某些排序约束来执行。 此外,由于解决微分问题时对数值精度的要求,模板处理的数据往往是高精度的浮动数据,这会消耗更多的片上内存用于平铺技术。 由于这些差异,模板往往会引发与卷积不同的优化。

\subsection{背景}