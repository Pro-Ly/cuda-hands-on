\section{卷积:常量内存和缓存简介}
在接下来的几章中,我们将讨论一组重要的并行计算模式。 这些模式是许多并行应用中出现的各种并行算法的基础。 我们将从卷积开始,它是一种流行的数组运算,在信号处理、数字记录、图像处理、视频处理和计算机视觉中以多种形式使用。 在这些应用领域中,卷积通常作为滤波器来执行,将信号和像素转换为更理想的值。 我们的图像模糊内核就是这样一个滤波器,它可以平滑信号值,以便人们可以看到大图趋势。 又例如,高斯滤波器是卷积滤波器,可用于锐化图像中对象的边界和边缘。

卷积通常执行大量算术运算来生成每个输出元素。 对于高清图像和视频等输出元素(像素)较多的大型数据集,计算量可能会很大。 一方面,卷积的每个输出数据元素可以彼此独立地计算,这是并行计算的理想特性。 另一方面,在处理具有一定挑战性的边界条件的不同输出数据元素时存在大量输入数据共享。 这使得卷积成为复杂的平铺方法和输入数据分级方法的重要用例,这也是本章的重点。

\subsection{背景}

\subsection{总结}
在本章中,我们研究了卷积作为一种重要的并行计算模式。 虽然卷积被用于许多应用,例如计算机视觉和视频处理,但它也代表了一种通用模式,构成了许多并行算法的基础。 例如,我们可以将偏微分方程求解器中的模板算法视为卷积的一种特殊情况; 这将是第 8 章“模板”的主题。 再例如,也可以将网格点力或势值的计算视为卷积的一种特例,这将在第17章“迭代磁共振成像重建”中介绍。 我们还将在第 16 章“深度学习”中应用我们在本章中学到的有关卷积神经网络的大部分知识。

我们提出了一种基本的并行卷积算法,其实现将受到访问输入和滤波器元素的 DRAM 带宽的限制。 然后,我们引入了常量内存并对内核和主机代码进行了简单修改,以利用常量缓存并消除对过滤器元素的几乎所有 DRAM 访问。 我们进一步引入了平铺并行卷积算法,该算法通过利用共享内存来减少 DRAM 带宽消耗,同时引入更多的控制流发散和编程复杂性。 最后,我们提出了一种平铺并行卷积算法,该算法利用 L1 和 L2 缓存来处理晕单元。

我们分析了平铺在提高算术与全局内存访问比率方面的优势。 分析是一项重要技能,有助于理解平铺对其他模式的好处。 通过分析,我们可以了解小图块尺寸的限制,这对于大滤波器和 3D 卷积尤其明显。

尽管我们仅展示了 1D 和 2D 卷积的内核示例,但这些技术也直接适用于 3D 卷积。 一般来说,由于维度较高,输入和输出数组的索引计算更加复杂。 此外,每个线程都会有更多的循环嵌套,因为在加载图块和/或计算输出值时需要遍历多个维度。 我们鼓励读者完成这些高维内核作为家庭作业。