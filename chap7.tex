\section{卷积:常量内存和缓存简介}
在接下来的几章中,我们将讨论一组重要的并行计算模式。 这些模式是许多并行应用中出现的各种并行算法的基础。 我们将从卷积开始,它是一种流行的数组运算,在信号处理、数字记录、图像处理、视频处理和计算机视觉中以多种形式使用。 在这些应用领域中,卷积通常作为滤波器来执行,将信号和像素转换为更理想的值。 我们的图像模糊内核就是这样一个滤波器,它可以平滑信号值,以便人们可以看到大图趋势。 又例如,高斯滤波器是卷积滤波器,可用于锐化图像中对象的边界和边缘。

卷积通常执行大量算术运算来生成每个输出元素。 对于高清图像和视频等输出元素(像素)较多的大型数据集,计算量可能会很大。 一方面,卷积的每个输出数据元素可以彼此独立地计算,这是并行计算的理想特性。 另一方面,在处理具有一定挑战性的边界条件的不同输出数据元素时存在大量输入数据共享。 这使得卷积成为复杂的平铺方法和输入数据分级方法的重要用例,这也是本章的重点。

\subsection{背景}