\section{图遍历}
图是表示实体之间关系的数据结构。 涉及的实体表示为顶点,关系表示为边。 许多重要的现实世界问题自然地被表述为大规模图问题,并且可以从大规模并行计算中受益。 突出的例子包括社交网络和行车路线图服务。 并行化图计算有多种策略,其中一些以并行处理顶点为中心,而另一些则以并行处理边为中心。 图本质上与稀疏矩阵相关。 因此,图计算也可以用稀疏矩阵运算来表示。 然而,人们通常可以通过利用特定于所执行的图计算类型的属性来提高图计算的效率。 在本章中,我们将重点讨论图搜索,这是许多现实世界应用程序的基础的图计算。

\subsection{背景}

\subsection{总结}
在本章中,我们以广度优先搜索为例,了解了与并行图计算相关的挑战。 我们首先简要介绍了图的表示。 我们讨论了以顶点为中心和以边为中心的并行实现之间的差异,并观察了它们之间的权衡。 我们还看到了如何通过利用边界来消除冗余工作,并通过私有化来优化边界的使用。 我们还简要讨论了其他高级优化,以减少同步开销并改善负载平衡。